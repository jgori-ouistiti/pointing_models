\documentclass{article}
\usepackage{amsmath}
\usepackage{amssymb}
\usepackage{here}
\usepackage[all]{foreign}
\usepackage{graphicx}
\usepackage[colorlinks=true, allcolors=blue]{hyperref}
\title{Cover Letter for: Pointing models for users operating under different speed accuracy strategies}
\author{Julien Gori}
\begin{document}
\maketitle


Dear Editor of TOCHI, 

\vspace{2\baselineskip}


I am happy to submit this paper which presents a tentatively complete description of movement time distributions produced by participants during a pointing experiment. Compared to existing models such as Fitts' law, the emphasis here is on predicting both the movement time distribution, the distribution of the index of difficulty, and how both depend on each other, including when participants follow particular strategies (eg., emphasizing speed or accuracy). We present an empirical analysis that relies on data from four different controlled studies, conducted with three different protocols. The empirical analysis is also accompanied by a short theoretical analysis. I believe this analysis will allow us HCI researchers to better understand pointing, including in more ``natural'', outside the lab contexts.
A second part of the work is in providing models to generate movement time distributions. I namely show how the previous results can be exploited to suggest 3 candidate procedures to generate synthetic pointing data, which are evaluated on several criterion. 
The relationship between this paper and existing work, as well as the submission history of the paper, is detailed in the following statement of research.

\noindent\rule{\textwidth}{0.4pt}\\
\textbf{Statement of research:}\\
This paper has previously been submitted as a "short" paper at CHI 25. 1AC's meta-review read:


\textit{This work proposes new models that leverage copulas for pointing when users operate under different speed-accuracy strategies, but which take into account less optimistic models compared to prior work. While this approach has the potential to provide a significant contribution to the HCI community using an interesting approach, reviewers felt that the paper is difficult to understand and requires additional work to evaluate and validate the proposed model. Unfortunately, given the potential scope of work required to satisfy these issues, in discussion it was decided that one round of revise and resubmit (R\&R) would likely not be enough. As a result, this paper will not be considered in the R\&R round.}

\textit{To expand on the main issues discussed, all felt that the paper would benefit from clearer explanations and inclusion of intermediate steps to guide the reader in the body of the main paper itself. It is currently a short paper, however reviewers felt it was too short to properly explain the concepts and justifications and the inclusion of additional material would have significantly helped. I appreciate this is a complex topic, however all reviewers struggled to follow some of the topics and I think it important as many of the community as possible understands and appreciates this work. In order to evaluate and properly validate the proposed models, reviewers felt that the paper needs additional experiments and/or analysis that better reflect the situations that these models would be most applicable.}

I have tried to account for all of the good points made by the reviewers; as a result this has increased the page count by quite a large amount, and I couldn't possibly list all the changes made. Overall, I have provided more details on the technical parts of the manuscript, especially the theoretical proofs and the methods around copulas and EMG. I have also analyzed more diverse data than before, including the analysis of two new, previously published datasets. I believe this addresses the main criticisms raised.

This paper is not an extension of existing papers. However, it builds on several published works, a few of which I am co-author. Rest assured that the content of this paper is novel.


Thank you for considering this manuscript. \\
\begin{flushright}
Sincerely,\\
Julien Gori
\end{flushright}
\end{document}