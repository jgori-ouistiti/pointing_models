\documentclass{article}
\usepackage{amsmath}
\usepackage{amssymb}
\usepackage{here}
\usepackage[all]{foreign}
\usepackage{graphicx}
\usepackage[colorlinks=true, allcolors=blue]{hyperref}
\title{Cover Letter for: Pointing models for users operating under different speed accuracy strategies}
\author{Julien Gori}
\begin{document}
\maketitle


Dear Editor of TOCHI, 

\vspace{2\baselineskip}


I am glad to submit this paper which presents a tentatively complete description of movement time distributions produced by participants of a pointing experiment, including when they operate under various strategies. The empirical analysis relies on data from four different controlled studies, conducted with three different protocols. The empirical analysis is also accompanied by a short theoretical analysis. I believe this part of the work will allow us HCI researchers to better understand pointing.
A second part of the work is in providing models to generate movement time distributions. I namely show how the previous results can be exploited, and suggest 3 candidate procedures, which are evaluated on several criterion. 
The relationship between this paper and existing work is detailed in the following statement of research.

\noindent\rule{\textwidth}{0.4pt}\\
\textbf{Statement of research:}\\
This paper has previously been submitted as a "short" paper at CHI 25. 1AC's meta-review read:


\textit{This work proposes new models that leverage copulas for pointing when users operate under different speed-accuracy strategies, but which take into account less optimistic models compared to prior work. While this approach has the potential to provide a significant contribution to the HCI community using an interesting approach, reviewers felt that the paper is difficult to understand and requires additional work to evaluate and validate the proposed model. Unfortunately, given the potential scope of work required to satisfy these issues, in discussion it was decided that one round of revise and resubmit (R\&R) would likely not be enough. As a result, this paper will not be considered in the R\&R round.}

\textit{To expand on the main issues discussed, all felt that the paper would benefit from clearer explanations and inclusion of intermediate steps to guide the reader in the body of the main paper itself. It is currently a short paper, however reviewers felt it was too short to properly explain the concepts and justifications and the inclusion of additional material would have significantly helped. I appreciate this is a complex topic, however all reviewers struggled to follow some of the topics and I think it important as many of the community as possible understands and appreciates this work. In order to evaluate and properly validate the proposed models, reviewers felt that the paper needs additional experiments and/or analysis that better reflect the situations that these models would be most applicable.}

I have tried to account for all criticism raised by the reviewers; as a result this has increased the page count by a lot, and I couldn't list all the changes made, but overall I have provided more details on the technical parts of the manuscript and have analyzed more diverse data than before.


Thank you for considering this manuscript. \\
\begin{flushright}
Sincerely,\\
Julien Gori
\end{flushright}
\end{document}